\documentclass[../Ansoegning.tex]{subfiles}
\begin{document}
\begin{center}
    \Large{\textbf{Lægeerklæring}}
\end{center}

Jeg må på ærlig vis indrømme, at jeg var i tvivl om, hvorvidt en lægeerklæring var nødvendig, hvis man søger et legat hos \FondNavn, hvis man er en ung studerende som skal på udlandsophold. Men det har desværre ikke været muligt for mig, at anskaffe en lægeerklæring. 

Dette skyldes, at jeg på nuværende tidspunkt er bosat i Frankrig i forbindelse med mit praktikophold på \textit{Den Europæiske Organisation for Højenergifysik (CERN)}. Mit praktikophold i CERN har en varighed af seks måneder i perioden 1. februar til 31. juli 2019. Da jeg først opdagede at, {\FondNavn} støtter studerende på udlandsophold umiddelbart efter jeg var taget afsted, har det derfor ikke været muligt at anskaffe en lægeerklæring. 

Jeg har dog telefonisk været i dialog med min læge i Danmark, som efterspurgte yderligere oplysninger vedrørende formålet med lægeerklæringen. Ifølge min læge, udarbejdes en lægeerklæring for at dokumentere ét eller flere veldefinerede formål. Da jeg ikke har været i stand til at finde disse oplysninger på fondens hjemmeside, er dette blot endnu en årsag, til at jeg ikke har været i stand til anskaffe en lægeerklæring.

Men min læge kan bekræfte, at jeg har været i dialog med dem, hvis I ønsker at kontakte dem. Min læge er som følger:
\begin{itemize}
    \item[] Lægerne Sankt Olufsgade
    \item[] Sankt Olufs Gade 1     
    \item[] 8000 Aarhus C           
    \item[] Telefon: 86 12 08 77
\end{itemize}

Jeg håber, at I vil tage min anmodning til efterretning, på trods af denne åbenlyse mangel i min ansøgning. Men jeg kan dog informere Jer om, at jeg er sund og rask, samt dyrker motion tre gange dagligt. Årsagen til at jeg søger et legat, skyldes udelukkende de økonomiske og sociale forhold, som jeg har dokumenteret i min ansøgning.

\end{document}