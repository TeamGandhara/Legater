\documentclass[../Ansoegning.tex]{subfiles}
\begin{document}
%--------------------Overskrift-------------
\par{\centering
		{\Large \textbf{Redegørelse for sociale og økonomiske forhold}} \par}
		%hrule\vspace{-1mm}

\textbf{Kære legatadministrator hos \textit{\FondNavn},}

Jeg er en dansk statsborger med rødder i Afghanistan. Jeg kom til Danmark som syvårig i år 2002 med min mor og lillesøster. Få måneder senere kom min far.

Mine forældres udvandring er en kulmination af borgerkrigen i Afghanistan, den resulterende
frygt og usikkerhed heraf, samt min mors sygdom. Siden hun var en lille pige, har min mor
lidt under, at have kronisk nedsat nyrefunktion. På daværende tidspunkt kunne dette ikke
helbredes i Afghanistan. Da min far ikke kunne se en fremtid for hverken min mor eller sine
børn, måtte de søge lykken andre steder.

Jeg ankom til Sandholmlejren i Birkerød i vinteren år 2002. Her boede jeg i få måneder
med min mor og lillesøster. Jeg husker det som en meget mørk og ensom tid, idet vi ikke
kendte nogen eller havde nogen form for beskæftigelse. Herefter blev vi flyttet til Frederikshavn
Asylcenter, hvorved min mor kunne blive behandlet på Aalborg Universitetshospital.
Vi boede i asylcenteret i 2,5 år, hverefter vi fik asyl. Jeg husker min tid i Frederikshavn som
det lykkeligste, da det var her min barndom startede: her kunne jeg lege med andre børn
med samme baggrund og vi havde ingen bekymringer: verden gik ikke længere ud end til
asylcenterets hegn.

Vi flyttede til Nykøbing Mors efter vi fik asyl, idet det stadig var relativt tæt på Aalborg, til at
min mor kunne få sin behandling. Her begyndte jeg i skole for første gang i modtagerklassen.
Efter blot to måneder kunne jeg sproget nok, til at jeg kunne starte i 4. klasse med de andre
’almindelige’ elever.

Siden en meget tidlig alder, har jeg været bevidst omkring, at intet i livet kommer af sig selv.
Succes opnår man kun gennem hårdt arbejde. Dette er en tankegang jeg har arvet fra mine
forældre: uanset hvor tynget de har været af sygdom, bekymringer for børn og egen familie
i Afghanistan, usikkerheder omkring fremtiden i et fremmede land, har de altid opretholdt
en høj pande og rank ryg - for at sætte et eksempel for deres børn.

Resultatet af dette har jeg ofte set: jeg gik ud af 9. klasse i med klassens højeste gennemsnit
og en personlig anerkendelse af skoleforstanderen, fordi jeg arbejdede hårdere end nogen
anden og klarede mig bedre end de fleste, på trods at min baggrund og kortere tid i skolen.
På samme vis dimitterede jeg fra gymnasiet med et af klassens højeste gennemsnit og en
anerkendelse af skolen vedr. min store indsats i min tid på gymnasiet.

Dette er fortsat indtil i dag, hvor jeg klarer mig rigtig godt på mit studie. Jeg besidder et af
klassens højeste gennemsnit på \avgKar. Jeg nyder en stor respekt blandt mine klassekammerater
og undervisere. Fordi jeg klarede mig særlig godt til ét af mine eksaminer på 3. semester,
tilbød skolen mig at blive studentermedhjælper, uden at jeg havde anmodet om det.

Min baggrund og egen indsats, kulminerer i, at jeg nu er ansat som ingeniørpraktikant ved \textit{Den Europæiske Organisation for Højenergifysik (CERN)} - en mulighed som kun de færreste og dygtigste realiserer. Men jeg vil gerne rage endnu højere op: jeg har længe haft en ambition om at studere på ét af verdens førende tekniske universiteter. Denne ambition kan jeg endelig realiserer gennem et udvekslingsophold på NTU i Singapore. Dette giver anledning til udgifter som er væsentlig højere end hvad jeg til dagligt er vant til. Jeg har selv arbejdet meget gennem tiderne, og min
arbejdende far har tilbudt at støtte mig med \FamSupp kr. om måneden under mit ophold.
Dette dækker dog ikke mit underskud, men han har heller ikke muligheden for at støtte
mere end det, da min mor er på førtidspension og hendes medicinudgifter er enorme.

Jeg håber derfor på at blive tildelt en legat fra \FondNavn, så jeg kan
gennemføre mit udlandsophold. Så kan jeg endelig give tilbage til det samfund som har givet
og formet mig så meget.

Jeg takker {\FondNavn } for denne enestående mulighed!

\end{document}